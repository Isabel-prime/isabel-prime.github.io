%%%%%%%%%%%%%%%%%%%%%%%%%%%%%%%%%%%%%%%%%
% Plasmati Graduate CV
% LaTeX Template
% Version 1.0 (24/3/13)
%
% This template has been downloaded from:
% http://www.LaTeXTemplates.com
%
% Original author:
% Alessandro Plasmati (alessandro.plasmati@gmail.com)
%
% License:
% CC BY-NC-SA 3.0 (http://creativecommons.org/licenses/by-nc-sa/3.0/)
%
% Important note:
% This template needs to be compiled with XeLaTeX.
% The main document font is called Fontin and can be downloaded for free
% from here: http://www.exljbris.com/fontin.html
%
%%%%%%%%%%%%%%%%%%%%%%%%%%%%%%%%%%%%%%%%%

%----------------------------------------------------------------------------------------
%	PACKAGES AND OTHER DOCUMENT CONFIGURATIONS
%----------------------------------------------------------------------------------------

\documentclass[a4paper,10pt]{article} % Default font size and paper size

\usepackage{fontspec} % For loading fonts
\defaultfontfeatures{Mapping=tex-text}
\setmainfont[SmallCapsFont = Fontin SmallCaps]{Fontin} % Main document font

\usepackage{xunicode,xltxtra,url,parskip} % Formatting packages

\usepackage[usenames,dvipsnames]{xcolor} % Required for specifying custom colors

\usepackage[margin=0.6in]{geometry}
%\usepackage[big]{layaureo} % Margin formatting of the A4 page, an alternative to layaureo can be \usepackage{fullpage}
% To reduce the height of the top margin uncomment: 
%\addtolength{\voffset}{-1.3cm}

\usepackage{hyperref} % Required for adding links	and customizing them
\definecolor{linkcolour}{rgb}{0,0.2,0.6} % Link color
\hypersetup{colorlinks,breaklinks,urlcolor=linkcolour,linkcolor=linkcolour} % Set link colors throughout the document

\usepackage{titlesec} % Used to customize the \section command
\titleformat{\section}{\Large\scshape\raggedright}{}{0em}{}[\titlerule] % Text formatting of sections
\titlespacing{\section}{0pt}{3pt}{3pt} % Spacing around sections

\usepackage{tabulary}

\begin{document}

\pagestyle{empty} % Removes page numbering

\font\fb=''[cmr10]'' % Change the font of the \LaTeX command under the skills section

%----------------------------------------------------------------------------------------
%	NAME AND CONTACT INFORMATION
%----------------------------------------------------------------------------------------

\par{\centering{\Huge Isabel Longbottom}\par} % Your name
\par{\centering{\textsc{email:} \href{mailto:ilongbottom@math.harvard.edu}{ilongbottom@math.harvard.edu}}\par}

%----------------------------------------------------------------------------------------
%	EDUCATION
%----------------------------------------------------------------------------------------

\section{Education}

\begin{tabular}{rl}	
\textsc{December} 2021 &  Bachelor of Philosophy - Science (Honours), Mathematics major\\& \normalsize\textbf{The Australian National University}, Canberra\\
&GPA of 7.00 (out of 7.00), First Class Honours\\

%------------------------------------------------

\textsc{} 2017 & Western Australian Certificate of Education\\&\textbf{Rossmoyne Senior High School}, Perth\\
& Beazley Medal, Dux, ATAR of 99.95\\
\end{tabular}

%----------------------------------------------------------------------------------------
%	RESEARCH INTERESTS
%----------------------------------------------------------------------------------------

\section{Research Interests}

I am interested in algebraic geometry, representation theory, the derived category, and other topics in algebra.

%----------------------------------------------------------------------------------------
%	SCHOLARSHIPS AND ADDITIONAL INFO
%----------------------------------------------------------------------------------------

\section{Awards and Scholarships}

\begin{tabular}{rl}

2022 & Attended the IAS/PCMI Summer Program \footnotesize on number theory informed by computation\\

2020 & Alan McIntosh Prize \footnotesize for third year mathematics, ANU \normalsize\\

2020 & Hanna Neumann Prize \footnotesize for third year mathematics, ANU \normalsize\\

2020 & Priscilla Fairfield Bok Prize \footnotesize for women in third year science courses, ANU \normalsize\\

2019 & Hanna Neumann Prize \footnotesize for second year mathematics, ANU \normalsize\\

2019 & Priscilla Fairfield Bok Prize \footnotesize for women in third year science courses, ANU \normalsize\\

2018 & Hanna Neumann Prize \footnotesize for first year mathematics, ANU \normalsize\\

2018 & Mavis Prater Prize \footnotesize for women in mathematics, ANU \normalsize\\

\textsc{July} 2017 & Tuckwell Scholarship \footnotesize for an undergraduate degree at the ANU\\

2015, 2016 \textsc{and} 2017 & Australian Mathematics Olympiad Camp Senior \footnotesize \normalsize\\

2015-2017 & Bronze (2015, 2016) and Silver (2017) awards, Australian Mathematics Olympiad\footnotesize \normalsize\\

\end{tabular}

%----------------------------------------------------------------------------------------
%	COMPUTER SKILLS 
%----------------------------------------------------------------------------------------

\section{Computer Languages/Tools}

(alphabetically) Ada, Haskell, Java, {\fb \LaTeX}, Lean, Macaulay2, Magma, Python, Sage

%----------------------------------------------------------------------------------------
%	WORK EXPERIENCE 
%----------------------------------------------------------------------------------------

\section{Teaching \& Leadership Experience}

\begin{tabular}{r|p{12cm}}

\textsc{January 2020, 2021, 2022} & Tutor at the National Maths Summer School \emph{}\\
& \footnotesize{Guided students to explore mathematical problems and critiqued their proofs.}\\
\multicolumn{2}{c}{} \\

%------------------------------------------------

\textsc{2020 and 2021} & Demonstrator for MATH1115, MATH1116, MATH2222, and MATH2322, ANU \emph{}\\
& \footnotesize{Ran (online and in-person) workshops, and marked assignments and exams for first- and second- year undergraduate courses in calculus, linear algebra, and group \& ring theory. This position is equivalent to a TA.}\\
\multicolumn{2}{c}{} \\

%------------------------------------------------

\textsc{December 2020} & Maths tutor for the Curious Minds STEM program, Online \emph{}\\
& \footnotesize{Ran maths problem-solving sessions for rural female-identifying high school students.}\\
\multicolumn{2}{c}{} \\

%------------------------------------------------

\textsc{September-November 2019} & Drop-in tutor, ANU \emph{}\\
& \footnotesize{Drop-in tutor for first year courses during the exam preparation period.}\\
\multicolumn{2}{c}{} \\

%------------------------------------------------

\textsc{2019} & Peer-Assisted Learning Mentor for MATH1115 and MATH1116, ANU \emph{}\\
& \footnotesize{The Peer-Assisted Learning program at the ANU employs second-year undergarduates to prepare activities for and teach tutorials to first-year students.}\\
\multicolumn{2}{c}{} \\

%------------------------------------------------

\textsc{January 2018} & Experienced Group at the National Maths Summer School, Canberra \emph{}\\
& \footnotesize{Explored topics in cryptography, algebraic curves, and algorithms with the aim of discovering results and proofs by hand. Organised social activities for younger students.}\\
\multicolumn{2}{c}{} \\

\end{tabular}

%----------------------------------------------------------------------------------------
%	RESEARCH PROJECTS
%----------------------------------------------------------------------------------------

\section{Research Projects \& Graduate Coursework}

\bgroup
\def\arraystretch{1.8}
\begin{tabulary}{\textwidth}{ C  C  L }
\textbf{Project Title} & \textbf{Supervisor} & \textbf{Description}\\
\hline

Derived Quiver Representations & Dr Asilata Bapat & \footnotesize This was my Honours project. I studied the bounded derived category of quiver representations for an acyclic quiver, and gave a proof that derived reflection functors are equivalences of triangulated categories. I also looked at the iterated weight filtration, which gives a canonical refinement of the Harder-Narasimhan filtration coming from a stability condition.
\normalsize\\

Crystals: Combinatorial Algorithms and Tensor Categories & Dr Noah White & \footnotesize In this lecture course, we developed the theory of braided and coboundary monoidal categories, and studied the induced actions by the braid and cactus groups. We also defined reflection groups and root systems before proving the classification of irreducible finite real reflection groups. We constructed crystals of semistandard tableaux for the ($GL_n$) root datum, and studied the RSK algorithm, Sch\"{u}tzenberger involution and Littlewood-Richardson coefficients.\normalsize\\

Algebraic \& Analytic Number Theory & Prof. Amnon Neeman & \footnotesize This reading course was devoted to various topics in modern number theory. The first half proved various standard results in algebraic number theory, such as the Kummer-Dedekind theorem, Minkowski’s bound and methods of computing the class group, working from Stevenhagen's \emph{Number Rings}. We then spent some time analysing techniques specific to Galois extensions of number fields. In the second half, we transitioned into analytic number theory, proving results such as the analytic class number formula. I gave a lecture presenting the proof of the prime number theorem.\normalsize\\

Riemann Surfaces & Dr Ian Le & \footnotesize This lecture course developed much of the important theory of Riemann surfaces, including the uniformization theorem, the existence of meromorphic functions on any compact Riemann surface, the classification of elliptic curves, and the Riemann-Roch formula. As a personal project, I wrote a short paper outlining the proof of Riemann-Roch for smooth 1-dimensional curves over finite fields. \normalsize\\

Three-Manifolds & Dr Joan Licata & \footnotesize This course was a broad survey of topics in 3-manifold theory, including Heegaard decompositions, (Dehn) surgery, Morse theory, Kirby calculus and more. As a personal project, I gave a talk and wrote a short paper on rational tangle replacement, and the equivalence between 2-fold cyclic branched covers of $S^3$ and manifolds obtained from $S^3$ via surgery on a strongly-invertible link. \normalsize\\

Foundations of Algebraic Geometry & Dr Anand Deopurkar & \footnotesize This reading course followed Ravi Vakil’s \emph{The Rising Sea: Foundations of Algebraic Geometry} to develop the theory of sheaves on schemes. The first half focused on the construction and basic properties of schemes and their morphisms, while the second half introduced quasicoherent and coherent sheaves, line bundles and sheaf cohomology. \normalsize\\

Vector Bundles \& K-Theory & Dr Vigleik Angeltveit & \footnotesize This course was based on Hatcher's book draft \emph{Vector Bundles \& K-Theory}. It developed the theory of vector bundles on a topological space, defining topological K-theory and working through some applications. As a personal project, I studied Schubert calculus and the cohomology of Grassmannians, including the Pieri and Giambelli formulas and Littlewood-Richardson coefficients. \normalsize\\

Computational Algebraic Geometry & Dr Martin Helmer & \footnotesize This course gave a computational approach to modern algebraic geometry, developing algorithms and showing their applications using the language Macaulay2. As a personal project, I studied tropical algebraic varieties from the book \emph{Introduction to Tropical Geometry} by Maclagan and Sturmfels, building towards the Structure Theorem.\normalsize\\

Perverse Sheaves \& Deligne-Lusztig Theory & Dr Asilata Bapat \& Dr Uri Onn & \footnotesize This formal lecture course was divided into two sections. In the perverse sheaves component, we developed the theory of derived categories and t-structures on a triangulated category. We used this to study perverse sheaves and the perverse t-structure induced by a perversity function. The Deligne-Lusztig component was devoted to various classifications for characters on finite reductive groups, including parabolic induction and Deligne-Lusztig induction. \normalsize\\

Calculus on Manifolds & Prof. Ben Andrews & \footnotesize I followed the text \emph{Calculus on Manifolds} by Michael Spivak to learn the elementary theory of differential geometry. Ben Andrews provided resources to extend the project into the topics of Riemannian metrics and de Rham Cohomology.\normalsize\\

Combinatorial Game Theory in Lean & Prof. Scott Morrison & \footnotesize This project was an attempt to formalise the basic theory of Conway's combinatorial games in the interative theorem proving language Lean. While this theory is mostly elementary, it interacted in surprising ways with the inductive type system in Lean, and the difficulty of this project lay in proving well-foundedness of the necessary recursive definitions in Lean by choosing the right data types.\normalsize\\
\end{tabulary}
\egroup
%----------------------------------------------------------------------------------------
\end{document}
